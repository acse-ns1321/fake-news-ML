
\documentclass[twoside,twocolumn]{article}

\usepackage{blindtext} % Package to generate dummy text throughout this template 

\usepackage[sc]{mathpazo} % Use the Palatino font
\usepackage[T1]{fontenc} % Use 8-bit encoding that has 256 glyphs
\linespread{1.05} % Line spacing - Palatino needs more space between lines
\usepackage{microtype} % Slightly tweak font spacing for aesthetics

\usepackage[english]{babel} % Language hyphenation and typographical rules

\usepackage[margin=0.5in,top=15mm,columnsep=10pt]{geometry} % Document margins
\usepackage[hang, small,labelfont=bf,up,textfont=it,up]{caption} % Custom captions under/above floats in tables or figures
\usepackage{booktabs} % Horizontal rules in tables

\usepackage{lettrine} % The lettrine is the first enlarged letter at the beginning of the text

\usepackage{enumitem} % Customized lists
\setlist[itemize]{noitemsep} % Make itemize lists more compact

\usepackage{abstract} % Allows abstract customization
\renewcommand{\abstractnamefont}{\normalfont\bfseries} % Set the "Abstract" text to bold
\renewcommand{\abstracttextfont}{\normalfont\small\itshape} % Set the abstract itself to small italic text

\usepackage{titlesec} % Allows customization of titles
\renewcommand\thesection{\Roman{section}} % Roman numerals for the sections
\renewcommand\thesubsection{\roman{subsection}} % roman numerals for subsections
\titleformat{\section}[block]{\large\scshape\centering}{\thesection.}{1em}{} % Change the look of the section titles
\titleformat{\subsection}[block]{\large}{\thesubsection.}{1em}{} % Change the look of the section titles

\usepackage{fancyhdr} % Headers and footers
\pagestyle{fancy} % All pages have headers and footers
\fancyhead{} % Blank out the default header
\fancyfoot{} % Blank out the default footer
\fancyhead[C]{Oil Data Analysis - Methodology $\bullet$ Apr 2022 } % Custom header text
\fancyfoot[RO,LE]{\thepage} % Custom footer text

\usepackage{titling} % Customizing the title section

\usepackage{hyperref} % For hyperlinks in the PDF

\usepackage{graphicx} %package to manage images
\usepackage{subfigure}
\usepackage{caption}
\usepackage{subcaption}
\usepackage{booktabs}
\usepackage{float}

\usepackage[table,xcdraw]{xcolor}
\graphicspath{ {./images/} }
%----------------------------------------------------------------------------------------
%	TITLE SECTION
%----------------------------------------------------------------------------------------

\setlength{\droptitle}{-4\baselineskip} % Move the title up

\pretitle{\begin{center}\Huge\bfseries} % Article title formatting
\posttitle{\end{center}} % Article title closing formatting
\title{DATA ANALYSIS METHODOLOGY} % Article title
\author{%
\textsc{NIRANJANA SUNDARARAJAN}
}
\date{\today} % Leave empty to omit a date
\renewcommand{\maketitlehookd}{%
% \begin{abstract}
% \noindent \blindtext % Dummy abstract text - replace \blindtext with your abstract text
% \end{abstract}
}

%----------------------------------------------------------------------------------------

\begin{document}

% Print the title
\maketitle

%----------------------------------------------------------------------------------------
%	ARTICLE CONTENTS
%----------------------------------------------------------------------------------------
\lettrine[nindent=2em,lines=1]{T}he following report highlight sthe methodology used for data analysis .\\


%------------------------------------------------
\section{METHODOLOGY}
\begin{enumerate}
\item Data Exploration(Exploratory Data Analysis)
\begin{itemize}
\item Graphical Analysis
\item Correlation - Test for spurious correlation(CCF Test)
\item Test for Stationarity - Augmented Dickey Fuller ( Is the distribution a Random Walk?)
\item Test for Seasonality 
\item Decompose Time Series Based on Seasonality
\item  BoxCox Method to make the function sattionary
\item Adjust for Seasonlity ( 
\end{itemize}
\item Feature Engineering
\item Feature Selection
\item Modelling/ Model Selection
\item  Model Testing/ Validation
\end{enumerate}



%----------------------------------------------------------------------------------------
%	REFERENCE LIST
%----------------------------------------------------------------------------------------
\begin{thebibliography}{99} 
\bibitem[1]{}abc
\newblock {\em abc}, .
 
 \bibitem[2]{}.
 \newblock {\emabc, abc.}


\end{thebibliography}

%----------------------------------------------------------------------------------------

\end{document}
